\begin{document}

This paper converts the description of the ChestX-ray8 dataset \cite{Wang2017} from the prose of its original paper into the structured format of a dataset datasheet \cite{Gebru2018}. 

ChestX-ray8 is a dataset with over 100,000 chest X-ray images and their labels. It was created and made publicly available by the National Institutes of Health Clinical Center \cite{Health2017}. It originally had images for eight diseases, enhanced later to cover fourteen diseases, resulting in the other name by which this dataset is known, ChestX-ray14. The paper describing the dataset still refers to it as \say{ChestX-ray8}.

Information for the datasheet was compiled from:

\begin{itemize}
    \item \href{https://arxiv.org/abs/1705.02315v5}{The latest (fifth) version of the paper} \cite{Wang2017}.
    \item \href{https://www.nih.gov/news-events/news-releases/nih-clinical-center-provides-one-largest-publicly-available-chest-X-ray-datasets-scientific-community}{The NIH news release} \cite{Health2017}.
    \item \href{https://lukeoakdenrayner.wordpress.com/2017/11/18/quick-thoughts-on-chestxray14-performance-claims-and-clinical-tasks/}{The initial review of the dataset by L. Oakden-Rayner} \cite{OakdenRayner2017}.
    \item \href{https://lukeoakdenrayner.wordpress.com/2017/12/18/the-chestxray14-dataset-problems/}{The detailed analysis of problems with the dataset by L. Oakden-Rayner} \cite{OakdenRayner2018}.
    \item \href{https://www.nature.com/articles/s41598-019-42294-8}{The study of of different CNNs using ChestX-ray8} \cite{Baltruschat2019}.
    \item \href{https://nihcc.app.box.com/v/ChestXray-NIHCC/file/219760887468}{The detailed description of the instances} \cite{HealthClinicalCenter2017}.
    \item \href{https://nihcc.app.box.com/v/ChestXray-NIHCC/file/220660789610}{The README file for the dataset} \cite{HealthClinicalCenter2017}.
    \item \href{https://nihcc.app.box.com/v/ChestXray-NIHCC/file/249502714403}{The FAQ for the dataset} \cite{HealthClinicalCenter2017}.
\end{itemize}

The datasheet is written from the first-person point of view, as if the authors had created it, to make it more realistic. Whenever applicable, the source for the information used in the datasheet is cited.

\end{document}
